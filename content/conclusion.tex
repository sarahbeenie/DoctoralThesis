\chapter{Conclusion and outlook}
\label{ch:conc}

In recent years, advanced technologies have led to the accumulation of vast quantities of biological data, in turn culminating in the paradoxical challenge of trying to ascertain how exactly each of these individual measured components fit together. By itself, this data is far too complex to interpret and meaningful patterns are essentially hidden from plain view. Moreover, many entities are engaged in complex interactions across biological scales that are overlooked in single \textit{omic} experiments. 

This dissertation has endeavoured to build a comprehensive picture of the mechanisms underlying normal and aberrant biological functioning by focusing on networks of interacting molecules across scales as the fundamental unit of study in lieu of individual molecules. Firstly, this has entailed bringing to light problems associated with pathway analysis for the interpretation of biological data and providing solutions to tackle these problems (Chapter \ref{ch:pathways}). Moreover, we have integrated interaction data dispersed across heterogeneous resources towards building more complete biological networks which we use in several applications presented in this dissertation. Secondly, we have complemented the analysis of gene expression data with protein-protein interaction networks as a potential avenue to generate hypotheses of novel, context-specific links and to re-define pathway boundaries by leveraging transcriptomic measurements (Chapter \ref{ch:contexts}). Thirdly, we operationalize biological networks for label propagation algorithms and for a novel pathfinding algorithm we have implemented for biomedical applications (Chapter \ref{ch:applications}).

The publications presented in Chapter \ref{ch:pathways} were primarily concerned with pathway analysis techniques that leverage pathway knowledge for the interpretation of high throughput data. Here, we demonstrated that despite the value of pathway enrichment analysis, various avenues to conduct such an experiment can lead to substantial variability in results. This variability can stem from various factors, such as interchangeable modular components, the choice of pathway database or gene set collection, as well as distinct features of different experimental datasets. These and other factors were the subject of the review presented in Section \ref{ch:review}. We especially focused on, i) supporting researchers in designing an experimental setup for this oft-used analysis which best reflects the research question at hand and, ii) advising researchers to bear in mind that modifications to the analysis can lead to varying results and interpretations, some of which may be inconsequential at best, and misleading at worst.

Then, in Section \ref{ch:pathwayforte} and \ref{ch:decopath}, we identified and tackled the problem of one major factor that contributes to variable results in pathway enrichment analysis, specifically, pathway database choice. In a benchmark study, we illustrated the critical importance of the careful consideration of this factor, demonstrating how different databases can yield disparate results. We also took a major step in consolidating several pathway databases such that researchers can simultaneously conduct multiple pathway analyses with different gene set collections and evaluate how results compare using the DecoPath web application we developed. Finally, we built an ontology which maps several pathway databases together, making it possible to investigate how alternative definitions of the same pathway can lead to differing results. 

While the association of genes and pathways to a phenotype is a valuable insight, this sort of analysis serves as just one component of a much broader investigation of experimental data. In Chapter \ref{ch:contexts}, we presented two publications aimed at contextualizing transcriptional patterns through a large-scale investigation of context-specific datasets. By charting gene expression patterns, we aimed at better understanding the pathophysiological mechanisms that lead to diseases or elucidate patterns which vary across different contexts. Specifically, publications presented in Sections \ref{ch:coxpath} and \ref{ch:contnext} have explored four biological contexts: i) diseases, ii) tissues, iii) cell types, and iv) cell lines. 

Across each of these contexts, we sought to divulge which patterns are uniquely characteristic to a given context, and which are recurrent across them. Furthermore, we sought to determine the degree to which edges in context-specific gene co-expression networks overlapped with known interactions in pathway and interaction databases. This was a challenging task given the constraints in overlaying these two disparate network types, and given that up to 55\% of protein interactions in PPI databases can be context-specific or transient \parencite{stacey2018}. Nonetheless, we were still able to highlight pathways that were highly similar to relevant context-specific networks. Finally, by deconstructing the patterns of each independent network, we endeavoured to shed light on biases and confounds which can occur due to overlapping contexts in experimental datasets, such as different compositions of cell types in tissues or diseases. 

Finally, in Chapter \ref{ch:applications}, we demonstrated how abstracting biological data as networks can make them amenable to network-based algorithms to ask questions such as, which genes with as of yet unknown or obscure functions could be associated with a given biological process, or which active modules or communities might be relevant to a particular disease. In Section \ref{ch:multipaths}, we introduced MultiPaths, a Python framework which provides implementations of several network diffusion algorithms for the analysis of multimodal biological networks. In line with the work presented in Section \ref{ch:pathwayforte}, case scenarios conducted in this publication revealed that larger and more comprehensive networks generated by consolidating multiple heterogeneous resources can be more robust for analyses than individual ones. 

Next, in Section \ref{ch:drug2ways}, we presented drug2ways, a novel pathfinding algorithm which explores all paths between pairs of nodes in multimodal networks for drug discovery. Here, we hypothesized that a network-based approach which considers the ensemble of paths between drug and disease-relevant nodes within a network is more intuitive and biologically meaningful than similar pathfinding approaches for drug discovery that leverage network proximity methods (e.g., shortest paths). Using drug-disease pairs in clinical trials, we were able to validate our approach, finding that the top drug-disease pairs prioritized by drug2ways included a large proportion of clinically investigated ones.

As biological networks have become more and more widely used to organize and formalize biological data, their infrastructure to enable algorithmic utility has also become more robust. Nonetheless, the inherent complexity of biological systems, the transient and context-specific nature of interactions, obstacles with regards to data availability and interoperability, and a static overview that is intended to explain what is fundamentally a dynamic system, are all major limitations which remain to be addressed. 

\section{Future outlook}

We foresee several future directions for the work that has been presented in this dissertation. One of the most pressing matters in relying upon pathways and networks for biological insights and biomedical applications is grappling with what is largely an incomplete picture of multimodal biomolecular interactions. These gaps in knowledge hold ramifications for our ability to piece together the mechanisms governing the biological processes we seek to investigate. Thus, first and foremost, we envisage the elucidation of more complete interactomes, in parallel to greater interoperability across heterogeneous resources which house their interactions. Nonetheless, recent advances in network biology are progressively seeing the development of more holistic, detailed and multi-modal networks constructed from heterogeneous resources, such as the PrimeKG to support precision medicine \parencite{chandak2022}. We anticipate future work will also expand and increasingly rely upon mappings across pathway databases, while the mappings we have created (see Section \ref{ch:decopath}) can lay the foundation for a larger and more inclusive pathway ontology. 

Within the context of this work, the addition of further pathway databases and mappings can facilitate pathway analysis by enabling a much broader coverage of pathway knowledge. More complete interactomes can also increase the capabilities of network-based algorithms to generate better predictions given that gaps in knowledge limit our abilities to accurately model biological mechanisms. Similarly, knowledge of all biomolecular interactions is crucial for applications which benefit from complementing knowledge and data-driven approaches, for example overlaying multi-\textit{omic} measurements with pathway knowledge or interaction networks for drug discovery. However, methodologies that integrate high dimensional data can be noisy and complex (e.g., a genomics experiment can generate over 1 terabyte (TB) of data in a single run \parencite{hasenauer2022}). Moreover, techniques that model such data, can be fraught with challenges that occur from seemingly finding meaningful patterns in the data, but which are instead confounding factors and basal levels of gene expression. 

Despite these challenges, future work focused on more granular investigations of co-expression data, such as single cell experiments, and the expansion to other contexts (e.g., species), can further help to differentiate between recurring and unique patterns. Additional lines of work that are crucial to understanding cellular functioning are considerations for not only individual cells, but also their local neighbourhood context. In particular, this can entail analyzing cell-cell interactions using the techniques presented in Chapter \ref{ch:contexts} that leverage PPI networks, pathway knowledge and data-driven co-expression networks, specifically from single-cell datasets. By taking gene expression measurements of secreted extracellular proteins and their membrane-bound receptor proteins, and overlaying these with PPI networks, sub-graphs of their local neighbourhoods that model intercellular signalling can be generated, especially for applications in medicinal biology. Finally, incorporating temporal dimensions using longitudinal data \parencite{bodein2022} and quantitative and dynamic modelling \parencite{villaverde2022} represent some of the next major frontiers in network biology.






