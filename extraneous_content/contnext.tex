\chapter{Elucidating context-specific expression patterns through network-based approaches}
%\chapter{Elucidating gene expression patterns across multiple biological contexts through a large-scale investigation of transcriptomic datasets}

\label{ch:contnext}

\section*{Summary}
 
This chapter presents the following publication \textbf{(see Appendix \ref{ap:contnext})}:

\begin{itemize}

\item[] Rebeca Queiroz Figueiredo, Sara Díaz del Ser, Tamara Raschka, Martin Hofmann Apitius, Alpha Tom Kodamullil, \textbf{Sarah Mubeen\footnotemark{}} and Daniel Domingo Fernández\footnotemark[\value{footnote}] (2022). Elucidating gene expression patterns across multiple biological contexts through a large-scale investigation of transcriptomic datasets. \textit{BMC Bioinformatics}, \textbf{FIXME}.

\item[] \textbf{Authors' contributions:} Daniel Domingo Fernández and Sarah Mubeen conceived and designed the study. Rebeca Queiroz Figueiredo and Tamara Raschka processed the transcriptomic datasets. Rebeca Queiroz Figueiredo implemented the methodology and analyzed the results supervised by Sarah Mubeen and Daniel Domingo Fernández. Sara Díaz del Ser implemented the web application. Rebeca Queiroz Figueiredo, Sarah Mubeen, and Daniel Domingo Fernández wrote the manuscript. Alpha Tom Kodamullil and Martin Hofmann Apitius reviewed the manuscript.

\end{itemize}

\footnotetext{Joint last authors.}

\noindent
In the previous chapter, we described a systematic approach for the study of disease-specific experimental datasets, identifying disease mechanisms through the integration of co-expression networks and pathway knowledge. Gene expression patterns are also observable in various other contexts, such as those responsible for normal physiology. Indeed, context-specific expression is often responsible for the diversity of functions and characterizations of cell types and tissues, each with their unique specializations. For instance, previous studies \parencite{dobrin2009, pierson2015, mckenzie2018} have analyzed gene expression patters at the cell and tissue level, finding significant cell and tissue type-specific expression signatures which aid in understanding human biology. Thus, in this publication, we subsequently expanded the scope of our analysis to include multiple additional contexts, namely, cell types, tissues, and cell lines. Our work demonstrates a hybrid approach for the analysis of experimental data to discern which signatures of co-expression networks may be of particular significance to a certain cell or condition and which are constant across them.

Together with the preceding publication, we have presented a large-scale investigation of gene expression patterns across multiple biological contexts. While the previous chapter focused on characterizing transcriptomic signatures pertaining to disease-specific co-expression networks, the work done here was concerned with multiple contexts within a normal physiological state (i.e., cell types, tissues and cell lines). In order to develop an overview of context-specific patterns that give rise to distinct biological processes, we collected over 600 experimental datasets and categorized them into nearly 100 sub-categories in one of the three studied contexts: cell types, tissues and cell lines. Retaining only the strongest pairwise correlations between genes, we constructed co-expression networks on which, analagous to the previous work, we conducted a series of analyses at the node, pathway and network levels. By once again leveraging a PPI network as a referential template, we identified the nodes which were most common across sub-categories as well as contexts and those which were already described in the literature. Finally, pathway and network based analyses relied upon the PPI network to map observed correlations within the experimental data to pathway knowledge embedded in the PPI network. In order to explore the networks generated in this work, we have made our findings freely available in a web application, along with data and scripts. 
