\chapter{Delineating major influential factors on pathway analysis}
%\chapter{On the influence of several factors on pathway enrichment analysis}

\label{ch:review}

\section*{Summary}

This chapter presents the following publication \textbf{(see Appendix \ref{ap:review})}:

\begin{itemize}

\item[] \textbf{Sarah Mubeen}, Alpha Tom Kodamullil, Martin Hofmann Apitius and Daniel Domingo Fernández (2022). On the influence of several factors on pathway enrichment analysis  \textit{Briefings in Bioinformatics}, 23:3.

\item[] \textbf{Authors' contributions:} Sarah Mubeen and Daniel Domingo Fernández wrote the manuscript. Alpha Tom Kodamullil and Martin Hofmann Apitius reviewed the manuscript.

\end{itemize}

\noindent
Among the techniques available for the interpretation of biological data, pathway enrichment analysis has emerged as one of the most prominent. The popularity of this type of analysis can be attributed to a more natural approach in which biological data is interpreted within a system-level context, guided by literature- and/or expert- based knowledge. Given its widespread popularity, several hundreds of enrichment methods and databases have been developed, yet paradoxically, gold standards have noticeably been lacking. Nonetheless, these methods and databases have been investigated in a number of benchmark studies alongside of various other factors to study the impact they produce on the results of enrichment analysis. 

This publication outlines a review study which covers key aspects of enrichment analysis and summarizes the results of studies which have evaluated their influence. Solutions to mitigate the effect of these factors and identify possible future benchmarks are also made.

In conclusion, it is apparent that, in many instances, results of enrichment analysis can be ascribed to various factors beyond the intended object of investigation. We have reviewed numerous studies which have demonstrated how alternative experimental designs of such an analysis can lead to different results. In doing so, we have provided a comprehensive overview of which aspects a researcher should be cognizant of in conducting an enrichment analysis and how these can impact results. Specifically, we have summarized the results of a dozen benchmark studies that have investigated the effect of variations on enrichment methods on results, finding severe inconsistencies across the performance of methods. In some extreme cases, different methods can go so far as to yield either all gene sets or no gene sets as significantly enriched on the same dataset and database, highlighting the importance of raising awareness of these factors. 

We have also critically reviewed several other major factors relating to the analysis, including variations in modular aspects of a typical enrichment analysis, gene set size, and gene set/pathway database choice, once again finding these to be highly consequential to the overall results. Finally, we have proposed possible future benchmarks, provided possible solutions to mitigate these factors, as well as guidelines to make well-informed decisions. 
