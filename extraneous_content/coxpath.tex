\chapter{Identifying disease mechanisms through integrated biological networks and expression data}
%\chapter{Towards a global investigation of transcriptomic signatures through co-expression networks and pathway knowledge for the identification of disease mechanisms}

\label{ch:coxpath}

\section*{Summary}

This chapter presents the following publication \textbf{(see Appendix \ref{ap:coxpath})}:

\begin{itemize}

\item[] Rebeca Queiroz Figueiredo, Tamara Raschka, Alpha Tom Kodamullil, Martin Hofmann Apitius, \textbf{Sarah Mubeen}\footnotemark{} and Daniel Domingo Fernández\footnotemark[\value{footnote}] (2021). Towards a global investigation of transcriptomic signatures through co-expression networks and pathway knowledge for the identification of disease mechanisms. \textit{Nucleic acid research}, 49(14):7939–7953.

\item[] \textbf{Authors' contributions:}. Martin Hofmann Apitius conceived the original idea. Daniel Domingo Fernández and Sarah Mubeen designed and supervised the study. Tamara Raschka implemented the pipeline to download, process and categorize the gene expression datasets. Rebeca Queiroz Figueiredo and Tamara Raschka generated the co-expression networks for each group. Sarah Mubeen and Daniel Domingo Fernández generated the interactome network. Rebeca Queiroz Figueiredo performed the analyses. Rebeca Queiroz Figueiredo, Tamara Raschka, Sarah Mubeen and Daniel Domingo Fernández interpreted the results. Rebeca Queiroz Figueiredo, Tamara Raschka, Sarah Mubeen and Daniel Domingo Fernández wrote the manuscript.

\end{itemize}

\footnotetext{Joint last authors.}

\noindent
In this chapter and the one that follows, we introduce a specific class of biological networks in which nodes (genes) are connected depending on the strength of their correlations. These networks, termed gene co-expression networks, are often used for the analysis of experimental datasets as they allow for the simultaneous analysis of thousands of genes and facilitate the deciphering of gene expression patterns across multiple conditions. In this publication, we demonstrate how, when these gene co-expression networks are superimposed with a protein protein interaction (PPI) network and pathway knowledge, one can connect the transcriptome with the proteome and contextualize gene co-expression patterns with prior, literature-based knowledge. Using a combined knowledge- and data- driven approach, we show how insights can be gained on common and unique mechanisms that underlie disease pathophysiology. 

In this work, we have systematically investigated over sixty diseases, studied in hundreds of experimental datasets in a joint, knowledge- and data- driven manner. We constructed disease-specific co-expression networks from transcriptomic datasets and used a PPI network as a template to contextualize  co-expression patterns that emerge across conditions. We conducted three different analyses of the co-expression networks, at the node, edge and pathway levels, each of which was complemented by a parallel investigation of a PPI network. Specifically, we first identified unique and common proteins across diseases in a node-level analysis and evaluated their consistency against nodes from the PPI network as a proxy for their coverage in the scientific literature. We then performed two subsequent analyses assuming that when a given pathway is relevant to a disease, the genes in the pathway will also tend to be strongly correlated in the disease-specific co-expression networks. in an edge-level analysis, we studied the presence of common disease-specific correlations within the PPI network. Finally, in a pathway-level analysis, we explored the consensus and/or disagreements between connections in the disease specific co-expression networks and known PPIs in biological pathways. 
